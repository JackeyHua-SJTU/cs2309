本项目的名字是画廊问题,核心目标是解决对于一个多边形画廊,
一个摄像头能覆盖的最大监控面积。作为拓展,我还考虑了两个摄像头、
多边形画廊中存在障碍物和输入的边集包括弧的情况。我将对每一种情况进行
问题陈述与分析。

首先是单个摄像头的问题。我的核心思路是遍历多边形内部的每个
像素点,判断它能够看到的顶点集,并且计算出当前像素点到每个能看到的顶点
的射线与多边形边集的交点,从而得到可以覆盖的多边形区域,最后计算面积。在这种情况下,需要解决的问题有判断一个像素点
是否在多边形内部,判断线段是否完全在多边形内部,获得射线与多边形的交点,对交点进行排序,计算多边形面积。
针对\textbf{判断点在形内的问题},我用射线法的方法,从点射出一条指向无限远的边,然后求其与多边形边集的交点个数,如果
是奇数,那么就是在形内,否则在形外。射线穿过顶点需要特别判断,如果该顶点是边的上顶点才会计算交点,如果是是下顶点则不会计算交点。
针对\textbf{判断线段是否完全在多边形内部},我的思路是先求得线段与边集的交点,将线段分成有限个子线段,然后判断每个子线段的中点是否在多边形内部。
针对\textbf{获得射线与多边形的交点},我的思路是遍历多边形的边集,求出每条边与射线的交点,并将其放入一个\texttt{std::set}中存储。
针对\textbf{对交点进行排序},因为所有的交点都是在多边形的边集上,所以只需要按顺时针或者逆时针的顺序遍历边集,然后就可以得到点集的排序。
针对\textbf{计算多边形面积},我利用叉乘计算三角形面积的方法,将多边形的每条边和原点构成一个三角形,遍历所有三角形,计算面积之和,再取绝对值就是多边形的面积。

其次是两个摄像头的问题。我的核心思路是遍历多边形内部的每个像素点,随机选取40个
不同于当前像素点的点,获得两个摄像头能覆盖区域的多边形的并集,计算面积。
在这种情况下,需要额外解决的问题有获取两个多边形的并集。在实现过程中,我使用了\texttt{boost::geometry::union_}函数。
这个函数的输入是两个多边形,输出的是多边形的并集,假如输入的两个多边形没有交集,那就回返回数个没有交集的多边形。然后只需要遍历返回的多边形,将其用\texttt{FLTK}可视化,计算面积之和(即并集的面积)。
之所以采用随机选40个点,而不是遍历其他点是因为在实际运行中,后者的运行时间非常的长,因此进行了一定的剪枝。

第三是存在障碍物的情况。解决的思路与单个摄像头的问题相似,但存在部分需要修改的地方。首先是判断点在多边形内部而不是障碍物内部,
其次是判断线段完全在多边形内部而不会进入障碍物的内部,接着是求摄像机与顶点的射线和边集的交点,最后是对于获得的与多边形边集和障碍物边集的交点集如何进行排序。
针对\textbf{判断点在内部}的问题,可以复用无障碍物情况下的代码并且进行两次判断,判断该点是否在没有障碍物情况下的多边形内,而且判断该点是否不在障碍物内即可。
针对\textbf{判断线段是否完全在多边形内部},需要在无障碍物的基础上额外计算线段与障碍物的交点,并且判断每个子线段的中点是否在多边形内而不在障碍物内,即调用当前情况下的判断点在内部的函数。
针对\textbf{求摄像机与顶点的射线和边集的交点},我的思路是遍历多边形的边集和障碍物的边集,求出每条边与射线的交点,并且判断交点与摄像机构成的线段是否在多边形内部,即调用了上一步所构造的函数,如果线段在内部,则将交点放入一个\texttt{std::set}中存储。
针对\textbf{对交点进行排序},我的观察是可覆盖的多边形的边要么与原多边形和障碍物的边集重合,要么是与相机与某个顶点的射线相重合。
因此我首先对交点集按照与摄像头的线段的斜率进行排序,相同斜率的情况下按照距离摄像头的距离进行排序。接着我们需要调整部分相同斜率的交点的顺序,
而其依据就是前一个斜率的最后一个交点和当前斜率的第一个交点构成的线段应该与边集重合。如此调整完之后就可以获得正确的交点集的顺序。

最后是输入边集包括弧的情况。我的思路是需要输入弧的起始点,结束点和控制点(圆心)。需要解决的问题是如何去表示弧。
我的解决方案是等弧度的在弧上取点,用密集的线段来模拟弧。